\documentclass{article}
\setlength{\oddsidemargin}{0.25 in}
\setlength{\evensidemargin}{-0.25 in}
\setlength{\topmargin}{-0.6 in}
\setlength{\textwidth}{6.5 in}
\setlength{\textheight}{8.5 in}
\setlength{\headsep}{0.75 in}
\setlength{\parindent}{0 in}
\setlength{\parskip}{0.1 in}

% ===== PACKAGES =====
\usepackage{amsmath,amssymb}
\usepackage{color}
\usepackage{subfigure}
\usepackage{mdframed}
\usepackage{changepage}
\newmdenv[
  topline=false,
  bottomline=false,
  skipabove=\topsep,
  skipbelow=\topsep
]{siderules}
\renewcommand{\abstractname}{}

% ===== Jed's Defined Stuff ======
\DeclareMathOperator*{\argmin}{arg\!\min}
\DeclareMathOperator*{\argmax}{arg\!\max}
\usepackage{siunitx}
\usepackage{enumitem} % used to make alphabetical lists instead of numbered ones

% ===== VARIABLES =====
\def \R{\mathbb{R}}
\def \Pr{\mathbb{P}}
\def \D{{\rm D}}
\def \N{{\rm N}}
\def \xx{{\boldsymbol{\rm x}}}
\def \y{{\rm y}}

% ===== HEADER BOX =====
\newcommand{\lecture}[2]{
\pagestyle{myheadings}
\thispagestyle{plain}
\newpage
\noindent
\begin{center}
\rule{\textwidth}{1.6pt}\vspace*{-\baselineskip}\vspace*{2pt} % Thick horizontal line
\rule{\textwidth}{0.4pt}\\[1\baselineskip] % Thin horizontal line
\vbox{\vspace{2mm}
\hbox to 6.28in { {\bf CS 760: Machine Learning} \hfill Spring 2024 }
\vspace{4mm}
\hbox to 6.28in { {\Large \hfill #1  \hfill} }
\vspace{4mm}
\hbox to 6.28in { {\scshape Authors:}  #2 \hfill }}
\vspace{-2mm}
\rule{\textwidth}{0.4pt}\vspace*{-\baselineskip}\vspace{3.2pt} % Thin horizontal line
\rule{\textwidth}{1.6pt}\\[\baselineskip] % Thick horizontal line
\end{center}
\vspace*{4mm}
}



% =============== DOCUMENT ===============
\begin{document}
\lecture{Homework 2: Linear Regression}{Jed Pulley}

\begin{center}
{\Large {\sf \underline{\textbf{DO NOT POLLUTE!}} AVOID PRINTING, OR PRINT 2-SIDED MULTIPAGE.}}
\end{center}

\section*{Problem 2.1}

To minimize the MSE of $\theta$, we computer our estimator $\hat{\theta}$:

\[\hat{\theta} = \underset{\theta}{\arg\min} \Vert \bf{\sf{y}}  - \bf{X}  \theta \Vert ^2_2 \]

Expanding that out, we find:

\begin{align*}
  \Vert \bf{\sf{y}}  - \bf{X}  \theta \Vert ^2_2 
  & = (\bf{\sf{y}} - \bf{X}  {\theta})^{\sf{T}}(\bf{\sf{y}} - \bf{X}  {\theta}) \\
  & = \bf{\sf{y}}^{\sf{T}}\bf{\sf{y}} - \bf{\sf{y}}^{\sf{T}}\bf{X}{\theta} - {\theta}^{\sf{T}}\bf{X}^{\sf{T}}\bf{\sf{y}} + {\theta}^{\sf{T}}\bf{X}^{\sf{T}}\bf{X}{\theta} \\
  & = \bf{\sf{y}}^{\sf{T}}\bf{\sf{y}} - 2{\theta}^{\sf{T}}\bf{X}^{\sf{T}}\bf{\sf{y}} + {\theta}^{\sf{T}}\bf{X}^{\sf{T}}\bf{X}{\theta}
\end{align*}

From there, we take the derivative w.r.t. $\theta$:

\begin{align*}
  \frac{d}{d\theta} = 2\bf{X}^{\sf{T}}\bf{X}{\theta} - 2\bf{X}^{\sf{T}}\bf{\sf{y}}
\end{align*}

After setting this derivative to zero and solving for $\theta$, we find that our solution is:

\[\hat{\theta} = (\bf{X}^{\sf{T}}\bf{X})^{-1}\bf{X}^{\sf{T}}\bf{\sf{y}}\]

\section*{Problem 2.2}

Since know that $\epsilon$ is distributed normally, we can use the general multivariate normal distribution formula to derive our expression for the MLE of $\theta^{*}$:

\begin{align*}
  \mathbb{P}(\sf{y}, \bf{X} | \theta,\Sigma^{*}) = \frac{1}{{2\pi^{\frac{n}{2}} | \Sigma^{*} | }}e^{-\frac{1}{2}(\sf{y} - \bf{X} \theta)^T\Sigma^{*-1}(\sf{y} - \bf{X} \theta)}
\end{align*}

To find the MLE, we first take the natural log, then we take the derivative and set it to zero. After that, we solve for $\hat{\theta}$:

\begin{align*}
  \hat{\theta} = (\bf{X}^{\sf{T}}\Sigma^{*-1}\bf{X})^{-1}\bf{X}^{\sf{T}}\Sigma^{*-1}\bf{\sf{y}}
\end{align*}

This is very similar to the solution of Problem 2.1 above, but it's weighted by the inverse covariance matrix $\Sigma^*$

\section*{Problem 2.3}

Thanks to the central limit theorem, we can assume that $\theta^*$ is distributed normally under large sample sizes. The mean of this distribution is found to be $\theta^*$, but our covariance matrix is a little trickier to find. We need to first compute the Fisher Information matrix, given by:

\[I(\theta^*) = - \mathbb{E}[\frac{\partial^2}{\partial \theta^2} ln ( \mathbb{P}(\sf{y}, \bf{X} | \theta,\Sigma^{*})) ] \]

Therefore, the distribution of $\theta^*$ can be shows as:

\[\theta^* \sim \mathcal{N}(\theta^*, I(\theta^*)^{-1})\]

\section*{Problem 2.4}

\section*{Problem 2.5}

\section*{Problem 2.6}

\section*{Problem 2.7}
\begin{enumerate}[label=(\alph*)]
  \item 
\end{enumerate}

\end{document} 
































