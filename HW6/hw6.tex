\documentclass{article}
\setlength{\oddsidemargin}{0.25 in}
\setlength{\evensidemargin}{-0.25 in}
\setlength{\topmargin}{-0.6 in}
\setlength{\textwidth}{6.5 in}
\setlength{\textheight}{8.5 in}
\setlength{\headsep}{0.75 in}
\setlength{\parindent}{0 in}
\setlength{\parskip}{0.1 in}

% ===== PACKAGES =====
\usepackage{amsmath,amssymb}
\usepackage{color}
\usepackage{subfigure}
\usepackage{mdframed}
\usepackage{changepage}
\newmdenv[
  topline=false,
  bottomline=false,
  skipabove=\topsep,
  skipbelow=\topsep
]{siderules}
\renewcommand{\abstractname}{}

% ===== VARIABLES =====
\def \R{\mathbb{R}}
\def \Pr{\mathbb{P}}
\def \D{{\rm D}}
\def \N{{\rm N}}
\def \xx{{\boldsymbol{\rm x}}}
\def \y{{\rm y}}




% ===== HEADER BOX =====
\newcommand{\lecture}[2]{
\pagestyle{myheadings}
\thispagestyle{plain}
\newpage
\noindent
\begin{center}
\rule{\textwidth}{1.6pt}\vspace*{-\baselineskip}\vspace*{2pt} % Thick horizontal line
\rule{\textwidth}{0.4pt}\\[1\baselineskip] % Thin horizontal line
\vbox{\vspace{2mm}
\hbox to 6.28in { {\bf CS 760: Machine Learning} \hfill Spring 2024 }
\vspace{4mm}
\hbox to 6.28in { {\Large \hfill #1  \hfill} }
\vspace{4mm}
\hbox to 6.28in { {\scshape Authors:}  #2 \hfill }}
\vspace{-2mm}
\rule{\textwidth}{0.4pt}\vspace*{-\baselineskip}\vspace{3.2pt} % Thin horizontal line
\rule{\textwidth}{1.6pt}\\[\baselineskip] % Thick horizontal line
\end{center}
\vspace*{4mm}
}

% ===== Jed's Defined Stuff ======
\DeclareMathOperator*{\argmin}{arg\!\min}
\DeclareMathOperator*{\argmax}{arg\!\max}
\usepackage{siunitx}
\usepackage{enumitem} % used to make alphabetical lists instead of numbered ones
\usepackage{mathtools}
\usepackage{graphicx}
\usepackage{caption}

% =============== DOCUMENT ===============
\begin{document}
\lecture{Homework 6: Frequentists vs Bayesians}{Jed Pulley}

\begin{center}
{\Large {\sf \underline{\textbf{DO NOT POLLUTE!}} AVOID PRINTING, OR PRINT 2-SIDED MULTIPAGE.}}
\end{center}

\section*{Problem 6.1. Frequentist (MLE)}
To find the MLE of $p*$, we first start with the likelihood function:

  \[\mathbb{P}(p*) = \prod^n_{i=1} (p^{\sf{x}_i} (1 - p)^{1 - \sf{x}_i})\]

Then we take the log of the likelihood function:
  \[ log(\mathbb{P}(p^*)) = logp \sum_{i=1}^{n} {\sf{x}_i} + log (1 - p) \sum_{i=1}^{n}(1 - {\sf{x}_i}) \]

Using our optimization 101 technique, we get the derivative and set it equal to zero:
  \[ p_{MLE} = \frac{1}{n} \sum_{i=1}^{n} \sf{x}_i \]

Which we recognize to just be the mean.

\section*{Problem 6.2. Bayesian (MAP)}
We use our MAP formula $p_{MAP} = \underset{p}{\arg\max} \mathbb{P}(X | p)$ as our starting point. Using Bayes Rule, we can rearrange it as such:

\[ p_{MAP} = \underset{p}{\arg\max} \frac{\mathbb{P}(X | p) \mathbb{P}(p)}{\mathbb{P}(X)} \]

Since we're maximizing for $p$, we can ignore the $\mathbb{P(X)}$ term as it doesn't depend on $p$:
\[ p_{MAP} = \underset{p}{\arg\max} \mathbb{P}(X | p) \mathbb{P}(p)\]

This is very similar to our MLE statement above, with the exception of our prior term $\mathbb{P}(p)$ which we're assuming is information gathered about a previous event. Notably, if there is no prior information, our MAP estimate is equal to our MLE.

Given our prior $\mathbb{P}(p)$ being modeled as $Beta(\alpha, \beta)$, we can show our prior below:

\[ \mathbb{P}(p) = \frac{\Gamma(\alpha + \beta)}{\Gamma(\alpha)\Gamma(\beta)}p^{\alpha - 1}(1 - p)^{\beta - 1} \]

Using the prior information, the likelihood of our data (we'll say $X = [\sf{x}_1, \sf{x}_2, ..., \sf{x}_N]^{\sf{T}}$) can be written as:

\[ \mathbb{P}(X|p) = p^{\sum_{i=1}^{N} {\sf{x}}_i} (1 - p)^{N - \sum_{i=1}^{N} {\sf{x}}_i} =  p^{1^{{\sf{T}}}{\sf{X}}} (1 - p)^{N-1^{{\sf{T}}}{\sf{X}}} \]

Simplifying and matching with the form of a Beta distribution, we find our MAP estimator to be:

\[ \hat{p}_{MAP} = \frac{\sum_{i=1}^{N} {\sf{x}_i} + \alpha - 1}{N + \alpha + \beta - 2} \]

\section*{Problem 6.3}
  \begin{enumerate}[label=(\alph*)]
    \item 
    \item 
\end{enumerate}

\section*{Problem 6.4}
  \begin{enumerate}[label=(\alph*)]
    \item 
    \item 
\end{enumerate}

\section*{Problem 6.5}

\end{document} 
